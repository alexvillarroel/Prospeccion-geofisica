\documentclass{article}
%%%%%%%%%%%%%%%%%%%%%%%%%%%%%%%%%%%%%%%%%%%%%%%%%%%%%%%%%%%%%%%%%%%%%%%%%%%%%%%

%%%%%%%%%%%%%%%%%%%%%%%%%%%%% Using Packages %%%%%%%%%%%%%%%%%%%%%%%%%%%%%%%%%%
\usepackage{geometry}
\usepackage{graphicx}
\usepackage{amssymb}
\usepackage{amsmath}
\usepackage{amsthm}
\usepackage{empheq}
\usepackage{mdframed}
\usepackage{booktabs}
\usepackage{lipsum}
\usepackage{enumitem}
\usepackage{graphicx}
\usepackage{color}
\usepackage{psfrag}
\usepackage{pgfplots}
\usepackage{bm}
\usepackage[spanish]{babel}
%%%%%%%%%%%%%%%%%%%%%%%%%%%%%%%%%%%%%%%%%%%%%%%%%%%%%%%%%%%%%%%%%%%%%%%%%%%%%%%

% Other Settings

%%%%%%%%%%%%%%%%%%%%%%%%%% Page Setting %%%%%%%%%%%%%%%%%%%%%%%%%%%%%%%%%%%%%%%
\geometry{a4paper,margin=1.2in}

%%%%%%%%%%%%%%%%%%%%%%%%%% Define some useful colors %%%%%%%%%%%%%%%%%%%%%%%%%%
\definecolor{ocre}{RGB}{243,102,25}
\definecolor{mygray}{RGB}{243,243,244}
\definecolor{deepGreen}{RGB}{26,111,0}
\definecolor{shallowGreen}{RGB}{235,255,255}
\definecolor{deepBlue}{RGB}{61,124,222}
\definecolor{shallowBlue}{RGB}{235,249,255}
%%%%%%%%%%%%%%%%%%%%%%%%%%%%%%%%%%%%%%%%%%%%%%%%%%%%%%%%%%%%%%%%%%%%%%%%%%%%%%%

%%%%%%%%%%%%%%%%%%%%%%%%%% Define an orangebox command %%%%%%%%%%%%%%%%%%%%%%%%
\newcommand\orangebox[1]{\fcolorbox{ocre}{mygray}{\hspace{1em}#1\hspace{1em}}}
%%%%%%%%%%%%%%%%%%%%%%%%%%%%%%%%%%%%%%%%%%%%%%%%%%%%%%%%%%%%%%%%%%%%%%%%%%%%%%%

%%%%%%%%%%%%%%%%%%%%%%%%%%%% English Environments %%%%%%%%%%%%%%%%%%%%%%%%%%%%%
\newtheoremstyle{mytheoremstyle}{3pt}{3pt}{\normalfont}{0cm}{\rmfamily\bfseries}{}{1em}{{\color{black}\thmname{#1}~\thmnumber{#2}}\thmnote{\,--\,#3}}
\newtheoremstyle{myproblemstyle}{3pt}{3pt}{\normalfont}{0cm}{\rmfamily\bfseries}{}{1em}{{\color{black}\thmname{#1}~\thmnumber{#2}}\thmnote{\,--\,#3}}
\theoremstyle{mytheoremstyle}
\newmdtheoremenv[linewidth=1pt,backgroundcolor=shallowGreen,linecolor=deepGreen,leftmargin=0pt,innerleftmargin=20pt,innerrightmargin=20pt,]{theorem}{Theorem}[section]
\theoremstyle{mytheoremstyle}
\newmdtheoremenv[linewidth=1pt,backgroundcolor=shallowBlue,linecolor=deepBlue,leftmargin=0pt,innerleftmargin=20pt,innerrightmargin=20pt,]{definition}{Definition}[section]
\theoremstyle{myproblemstyle}
\newmdtheoremenv[linecolor=black,leftmargin=0pt,innerleftmargin=10pt,innerrightmargin=10pt,]{problem}{Problem}[section]
%%%%%%%%%%%%%%%%%%%%%%%%%%%%%%%%%%%%%%%%%%%%%%%%%%%%%%%%%%%%%%%%%%%%%%%%%%%%%%%

%%%%%%%%%%%%%%%%%%%%%%%%%%%%%%% Plotting Settings %%%%%%%%%%%%%%%%%%%%%%%%%%%%%
\usepgfplotslibrary{colorbrewer}
\pgfplotsset{width=8cm,compat=1.9}
%%%%%%%%%%%%%%%%%%%%%%%%%%%%%%%%%%%%%%%%%%%%%%%%%%%%%%%%%%%%%%%%%%%%%%%%%%%%%%%

%%%%%%%%%%%%%%%%%%%%%%%%%%%%%%% Title & Author %%%%%%%%%%%%%%%%%%%%%%%%%%%%%%%%
\title{Certamen 1 Introducción a la prospección remota}
\author{Alex Villarroel Carrasco}
%%%%%%%%%%%%%%%%%%%%%%%%%%%%%%%%%%%%%%%%%%%%%%%%%%%%%%%%%%%%%%%%%%%%%%%%%%%%%%%

\begin{document}
\usetikzlibrary{positioning}
	\tikzset{every picture/.style={line width=0.75pt}}    
	\pagestyle{plain}
	\begin{flushleft}
		Departamento de Geofísica \hfill Introducción a prospección geofísica\\
		Facultad de Cs. Físicas y Matemáticas\\
		\underline{Universidad de Concepción}
	\end{flushleft}
	
	\begin{flushright}\vspace{-5mm}
		\includegraphics[height=1.5cm]{escudo.png}
	\end{flushright}
	
	\begin{center}\vspace{-1cm}
		\textbf{\large Terreno de introducción a prospección geofísica\\2022-10-03}\\
		{\textcolor{blue}{Alex Villarroel Carrasco}}\\
	\end{center}
	\rule{\linewidth}{0.1mm}
	\\
    \section*{Desembocadura Rio Bio Bio}
    \textbf{Conceptos}:
    \begin{itemize}
        \item Antearco: Zona entre la zona de subducción y el arco de la cordillera de los Andes
        \item Igneas: Se forman por la solidificacion de un magma
        \item Sedimentarias: A partir de sedimentos
        \item Metamorficas: A partir de otras rocas sometidas a altas temperaturas y presiones
        \item Prisma de acreción: Cinturón metamórfico pareado.
        \begin{itemize}
            \item Basamento metamorfico
            \item Paleozoico
            \item Serie Oriental( bajo p/t) y Occidental (alto p/t)
        \end{itemize}
    \end{itemize}
    Fueron deformadas bajo diferentes condiciones metamórficas durante la Orogenia Gondwánica o San Rafael(Carbonífero Superior- Pérmico Inferior)
    Compuesta por secuencia metapelitas(pizarras y filitas), metasamitas y esquistos
    \subsection*{Intrusivos Trásicos\\Plutón Hualpén(Triásico Superior)}
    Batolito costero: Es un grupo de plutones en la cordillera de la Costa de Chile Central, que aparece contiguamente desde los 33 S hasta los 38 S
    \par Es un remanente de los arcos volcanicos. Intruye al basamento metamorfico
    \subsection*{Depósitos Cuaternarios}
    \textbf{Depósitos no consolidados}
    Asociados al sistema sedimentario de la costa actual.
    \begin{itemize}
        \item Depósitos fluviales
        \item Depósitos litorales
        \item Terrazas marinas
    \end{itemize}
\end{document}